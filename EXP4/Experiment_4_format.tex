\documentclass[11pt]{article}
\usepackage[margin=1in]{geometry}
\usepackage{enumitem}
\usepackage{hyperref}
\usepackage{graphicx}
\usepackage{array}
\usepackage{multicol}
\usepackage{longtable}
\usepackage{titlesec}
\usepackage{amsmath}

\begin{document}

%==================================================
\begin{center}
    \large \textbf{Sri Sivasubramaniya Nadar College of Engineering, Chennai} \\
    (An Autonomous Institution Affiliated to Anna University) \\
    \vspace{0.3cm}
\end{center}

\begin{table}[!h]
\renewcommand{\arraystretch}{1.5}
\resizebox{\textwidth}{!}{%
\begin{tabular}{|l|cll|}
\hline
Degree \& Branch & \multicolumn{1}{c|}{B.E. Computer Science \& Engineering} & Semester & VI \\ \hline
Subject Code \& Name & \multicolumn{3}{c|}{UCS2612 -- Machine Learning Algorithms Laboratory} \\ \hline
Academic Year & 2025--2026 (Even) & Batch & 2023--2027 \\ \hline
\end{tabular}
}
\end{table}

\begin{center}
\textbf{Experiment 4: Binary Classification using Linear and Kernel-Based Models}
\end{center}

%==================================================
\section*{Objective}
To classify emails as spam or ham using Logistic Regression and Support Vector Machine (SVM) classifiers and to analyze the effect of hyperparameter tuning on classification performance.

%==================================================
\section*{Dataset}
The \textbf{Spambase} dataset contains numerical features extracted from email content and a binary label indicating spam or non-spam (ham).

\textbf{Dataset Links (for reference):}
\begin{itemize}
    \item Kaggle: \href{https://www.kaggle.com/datasets/somesh24/spambase}{https://www.kaggle.com/datasets/somesh24/spambase}
\end{itemize}

%==================================================
\section*{Theory Background}

\subsection*{1. Logistic Regression}
Logistic Regression is a probabilistic classification algorithm used for binary classification problems.  
It models the probability that a sample belongs to a particular class using the sigmoid function:

\[
P(y=1|\mathbf{x}) = \frac{1}{1 + e^{-(\mathbf{w}^T\mathbf{x} + b)}}
\]

A threshold (usually 0.5) is applied to convert probability into class labels.

\textbf{Loss Function:}  
Logistic Regression minimizes the \textbf{log-loss (cross-entropy loss)}, which penalizes incorrect predictions.

\subsection*{Regularization in Logistic Regression}
Regularization prevents overfitting by adding a penalty to large coefficients.

\begin{itemize}
    \item \textbf{L1 Regularization (Lasso):}  
    Encourages sparsity by shrinking some coefficients exactly to zero.  
    Useful for feature selection.
    
    \item \textbf{L2 Regularization (Ridge):}  
    Penalizes large weights but keeps all features.  
    Improves model generalization.
\end{itemize}

\subsection*{Logistic Regression Hyperparameters}
\begin{itemize}
    \item \textbf{C (Inverse Regularization Strength):}  
    Controls the trade-off between model complexity and regularization.
    \begin{itemize}
        \item Small $C$: Strong regularization, simpler model
        \item Large $C$: Weak regularization, complex model
    \end{itemize}

    \item \textbf{Solver:}
    \begin{itemize}
        \item \texttt{liblinear}: Suitable for small datasets; supports L1 and L2
        \item \texttt{saga}: Efficient for large datasets; supports L1 and L2
    \end{itemize}
\end{itemize}

%--------------------------------------------------
\subsection*{2. Support Vector Machine (SVM)}
Support Vector Machine is a margin-based classifier that finds an optimal hyperplane separating two classes by maximizing the margin between them.

\textbf{Key Idea:}  
Only a subset of training points (support vectors) define the decision boundary.

\subsection*{SVM Kernels}
\begin{itemize}
    \item \textbf{Linear Kernel}: Suitable for linearly separable data
    \item \textbf{Polynomial Kernel}: Captures polynomial relationships
    \item \textbf{RBF Kernel}: Handles complex, non-linear boundaries
    \item \textbf{Sigmoid Kernel}: Similar to neural network activation
\end{itemize}

\subsection*{SVM Hyperparameters}
\begin{itemize}
    \item \textbf{C}: Controls margin vs misclassification
    \begin{itemize}
        \item Small $C$: Wider margin, higher bias
        \item Large $C$: Narrow margin, lower bias
    \end{itemize}
    \item \textbf{$\gamma$}: Controls influence of a single training point
\end{itemize}

%==================================================
\section*{Hyperparameter Tuning}

\subsection*{Grid Search}
Grid Search exhaustively evaluates all combinations of predefined hyperparameters using cross-validation.

\subsection*{Randomized Search}
Randomized Search evaluates randomly sampled hyperparameter combinations and is computationally efficient.

\textbf{Note:} Students may use either method. If both are used, results must be compared.

%==================================================
\section*{Implementation Steps}
\begin{enumerate}[label=\arabic*.]
    \item Load the dataset.
    \item Preprocess the data:
    \begin{itemize}
        \item Handle missing values
        \item Standardize features
    \end{itemize}
    \item Perform Exploratory Data Analysis (EDA).
    \item Split the dataset into training and testing sets.
    \item Train baseline Logistic Regression.
    \item Tune Logistic Regression hyperparameters.
    \item Train SVM with different kernels.
    \item Tune SVM hyperparameters.
    \item Evaluate models using standard metrics.
    \item Perform 5-Fold Cross-Validation.
\end{enumerate}

%==================================================
\section*{Hyperparameter Search Space}

\subsection*{Logistic Regression}
\begin{itemize}
    \item Regularization: L1, L2
    \item $C \in \{0.01, 0.1, 1, 10, 100\}$
    \item Solver: liblinear, saga
\end{itemize}

\subsection*{Support Vector Machine}
\begin{itemize}
    \item Kernel: Linear, Polynomial, RBF, Sigmoid
    \item $C \in \{0.1, 1, 10, 100\}$
    \item $\gamma \in \{\text{scale}, \text{auto}\}$
    \item Degree (Polynomial): $\{2,3,4\}$
\end{itemize}

%==================================================
\section*{Hyperparameter Tuning Results}
\begin{table}[h!]
\centering
\renewcommand{\arraystretch}{1.3}
\begin{tabular}{|l|c|c|c|}
\hline
Model & Search Method & Best Parameters & Best CV Accuracy \\ \hline
Logistic Regression & Grid / Random &  &  \\
SVM & Grid / Random &  &  \\ \hline
\end{tabular}
\end{table}

%==================================================
\section*{Logistic Regression Performance}
\begin{table}[h!]
\centering
\renewcommand{\arraystretch}{1.3}
\begin{tabular}{|l|c|}
\hline
Metric & Value \\ \hline
Accuracy &  \\
Precision &  \\
Recall &  \\
F1 Score &  \\
Training Time (s) &  \\ \hline
\end{tabular}
\end{table}

%==================================================
\section*{SVM Kernel-wise Performance}
\begin{table}[h!]
\centering
\renewcommand{\arraystretch}{1.3}
\begin{tabular}{|l|c|c|c|}
\hline
Kernel & Accuracy & F1 Score & Training Time (s) \\ \hline
Linear &  &  &  \\
Polynomial &  &  &  \\
RBF &  &  &  \\
Sigmoid &  &  &  \\ \hline
\end{tabular}
\end{table}

%==================================================
\section*{K-Fold Cross-Validation Results (K = 5)}
\begin{table}[h!]
\centering
\renewcommand{\arraystretch}{1.3}
\begin{tabular}{|c|c|c|}
\hline
Fold & Logistic Regression & SVM \\ \hline
Fold 1 &  &  \\
Fold 2 &  &  \\
Fold 3 &  &  \\
Fold 4 &  &  \\
Fold 5 &  &  \\ \hline
Average &  &  \\ \hline
\end{tabular}
\end{table}

%==================================================
\section*{Comparative Analysis}
\begin{table}[h!]
\centering
\renewcommand{\arraystretch}{1.3}
\begin{tabular}{|l|c|c|}
\hline
Criterion & Logistic Regression & SVM \\ \hline
Accuracy &  &  \\
Model Complexity & Low & High \\
Training Time & Low & High \\
Interpretability & High & Low \\ \hline
\end{tabular}
\end{table}

%==================================================
\section*{Observations}
\begin{itemize}
    \item Identify the best-performing classifier.
    \item Explain the impact of regularization.
    \item Discuss kernel behavior in SVM.
    \item Comment on bias--variance trade-off.
\end{itemize}

%==================================================
\section*{Learning Outcomes}
\begin{itemize}
    \item Understand probabilistic and margin-based classifiers.
    \item Apply hyperparameter tuning.
    \item Evaluate classification models.
    \item Interpret experimental results.
\end{itemize}

%==================================================
\section*{References}
\begin{itemize}
    \item \href{https://scikit-learn.org/stable/modules/linear_model.html#logistic-regression}{Scikit-learn: Logistic Regression}
    \item \href{https://scikit-learn.org/stable/modules/svm.html}{Scikit-learn: Support Vector Machines}
    \item \href{https://scikit-learn.org/stable/modules/grid_search.html}{Scikit-learn: Hyperparameter Optimization}
    \item \href{https://www.kaggle.com/datasets/somesh24/spambase}{Spambase Dataset – Kaggle}
    \item \href{https://archive.ics.uci.edu/ml/datasets/spambase}{UCI ML Repository – Spambase}
\end{itemize}

\end{document}
